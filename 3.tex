\documentclass{beamer}
\usepackage{ngerman}
\usepackage{listings}
\usepackage {ulem}
\usetheme{Madrid}
\title{Programmieren Tutorium}
\author{Florian Tobias Schandinat}
\date{8.11.2010}
\institute{FTS}
\definecolor{mygreen}{RGB}{0,128,0}
\definecolor{mybrown}{RGB}{128,0,0}
\definecolor{mygray}{RGB}{240,240,240}
\lstset{language=Java,
	backgroundcolor=\color{mygray},
	basicstyle=\tiny\ttfamily,
	keywordstyle=\color{blue},
	stringstyle=\color{mybrown},
	commentstyle=\color{gray},
	numbers=left,
	numberstyle=\color{mygreen},
	frame=single,
	tabsize=4,
	showstringspaces=false,
	xleftmargin=3.5em,
	xrightmargin=2em}


\begin{document}


\begin{frame}
\frametitle{Willkommen}
\pause
\begin{alertblock}{Letztes Mal}
\begin{itemize}
\item Variablen\\
\item Einfache Operationen\\
\item Methoden - Crashkurs
\end{itemize}
\end{alertblock}

\pause

\begin{block}{Dieses Mal}
\begin{itemize}
\item Konstruktoren und \texttt{final}\\
\item Programmablauf\\
\item Methoden -- Vertiefung\\
\item Geheimnisprinzip -- \texttt{public}, \texttt{protected}, \texttt{private}
\end{itemize}
\end{block}

\pause

\begin{exampleblock}{N"achstes Mal}
\begin{itemize}
\item Kontrollfluss\\
\item Tests
\end{itemize}
\end{exampleblock}
\end{frame}


\begin{frame}
\frametitle{Fazit: 1. "Ubungsblatt}
\begin{block}{Hinweise}
\begin{itemize}
\item Es ist immer besser, wenn eure L"osung kompiliert
\item Bezeichner sollten sorgf"altig gew"ahlt werden
\end{itemize}
\end{block}

\pause

\begin{block}{Musterl"osung}
inklusive JavaDoc
\end{block}
\end{frame}


\begin{frame}
\frametitle{Konstruktoren -- Einleitung}
\begin{block}{Wir finden Initialisierung toll...}
\pause
da Attribute und Variablen dann direkt den gewollten Wert haben\pause\\
Um dies auch f"ur komplexe Objekte erm"oglichen zu k"onnen haben wir Konstruktoren!\\[0.5em]
Beispiel: Initialisierung von komplexen Zahlen
\end{block}

\pause

\begin{block}{Konstruktoren -- "Uberblick}
\begin{itemize}
\item dienen dazu ein Objekt zu initialisieren\\
\item sind generell sehr "ahnlich zu Methoden\\
\item besitzen jedoch \alert{keinen R"uckgabetyp} und geben auch nichts zur"uck\\
\item k"onnen aber Parameter haben
\end{itemize}
\end{block}
\end{frame}


\begin{frame}[containsverbatim]
\frametitle{Konstruktoren -- Wie funktioniert das?}
\begin{block}{\texttt{new}}
\texttt{new} legt ein Objekt an und initialisert es mit dem passenden Konstruktor
Dieser wird ausgew"ahlt aufgrund
\begin{itemize}
\item der Klasse\\
\item der Parameteranzahl\\
\item der Reinfolge der Parametertypen
\end{itemize}
\alert{Parameternamen sind nicht relevant!}
\end{block}

\begin{block}{Beispiel}
\begin{lstlisting}
class Counter {
	int count;

	Counter() {
		count  = 1;
	}

	Counter(int startCount) {
		count = startCount;
	}
}
\end{lstlisting}
\end{block}
\end{frame}


\begin{frame}
\frametitle{\texttt{final}}
\begin{block}{Einleitung}
\texttt{final} dient zur Deklaration von Konstanten, die zur Laufzeit nicht ver"andert werden k"onnen\\
Es kann bei der Deklaration Attributen/Variablen vorangestellt werden\\
Wenn es in Verbindung mit \texttt{static} auftaucht, verwenden wir nur Gro"sbuchstaben f"ur den Bezeichner\\
Beispiele:\\
\lstinline|final static double PI = 3.14;|\\
\lstinline|final double epsilon = 1E-20;|
\end{block}

\pause

\begin{block}{\texttt{final} und Konstruktoren}
In Konstruktoren k"onnen (und m"ussen) konstante Attribute der Klasse gesetzt werden, sofern nicht bereits eine Zuweisung besteht
\end{block}
\end{frame}


\begin{frame}
\frametitle{"Ubung}
\begin{center}
\textbf{\Huge Auto (Aufgabe 1)}
\end{center}
\end{frame}


\begin{frame}
\frametitle{Geheimnisprinzip}
\begin{block}{Ziel}
Wiederverwendbarkeit und Wartbarkeit
\end{block}

\pause

\begin{block}{Lokalit"atsprinzip}
"Anderungen sollen nur lokale Auswirkungen haben
\end{block}

\pause

\begin{block}{Folgerung}
Daher erlauben wir nur Zugriff (von au"serhalb der Klasse) auf Attribute, Konstanten und Methoden wenn es erforderlich ist
\begin{description}
\item[\texttt{public}] Zugriff aus jeder Klasse m"oglich
\item[\texttt{protected}] Zugriff von innerhalb der Klasse und aus Unterklassen (sp"ater) erlaubt
\item[\texttt{private}] Zugriff nur von innerhalb der Klasse erlaubt
\end{description}
\end{block}
\end{frame}


\begin{frame}
\frametitle{"Ubung}
\begin{center}
\textbf{\Huge Auto (Aufgabe 2)}
\end{center}
\end{frame}


\begin{frame}
\frametitle{Ende}
\begin{block}{TODO}
\begin{enumerate}
\item Einreichen einer L"osung f"ur das 2. "Ubungsblatt im Praktomat bis \alert{22.11.2010, 13:00}
\item Anmelden f"ur den "Ubungsschein auf https://studium.kit.edu/ bis \alert{31.3.2011}
\end{enumerate}
\end{block}

\begin{block}{Vielen Dank f"ur die Aufmerksamkeit!}
...und viel Spa"s beim Programmieren :)
\end{block}
\end{frame}


\end{document}
