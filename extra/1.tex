\documentclass[a4paper]{article}
\usepackage{ngerman}
\usepackage[left=2.9cm,right=2.9cm,top=3cm,bottom=4.5cm]{geometry}


\begin{document}
\begin{center}
\textbf{\large Aufgabenblatt f"ur das 1. Vertiefungstutorium}
\end{center}

Im Folgenden wollen wir einen Taschenrechner, zum Beispiel in einer \texttt{class Calculator}, implementieren. Um uns dabei auch mit Stringfunktionen vertraut zu machen, wollen wir einen \textit{String} direkt parsen anstatt \textit{Integer.parseInt} zu benutzen.
\begin{enumerate}
\item Machen sie sich mit \textit{String.charAt} und \textit{String.length} vertraut (JavaDoc)

\item Schreiben sie eine Funktion \texttt{public Integer toInt(String number)}, die eine Ganzzahl einlesen kann
	\begin{itemize}
	\item Die Funktion soll sowohl positive als auch negative Zahlen einlesen k"onnen, wobei bei positiven das Vorzeichen optional sein soll
	\item Eine korrekte Zahl bestehe im Moment bis auf das Vorzeichen nur aus beliebig vielen Ziffern 0-9
	\item Machen sie sich Gedanken dar"uber, was passieren soll, wenn \textit{number} keine Zahl ist (Was ist der Unterschied zwischen \textit{Integer} und \textit{int}?)
	\end{itemize}

\item Testen sie ihre Methode \texttt{public Integer toInt(String number)}\\
Schreiben sie daf"ur eine \textit{main}-Methode. Versuchen sie dabei sowohl manuelles als auch automatisches Testen umzusetzen. Versuchen sie \glqq interaktives\grqq Testen, das kein neukompilieren ben"otigt um andere Zahlen zu testen, zu realisieren (Hinweis: \textit{args}).

\item Schreiben sie eine Funktion \texttt{public Integer calculate(String operation, Integer operand0, Interger operand1)} die das Ergebniss einer mathematischen Operation der Form \textit{operand0 operation operand1} berechnet und zur"uckgibt\\
Implementieren sie die Operationen +, --\\
Beachten sie die F"alle wenn die Operation oder ein Operand ung"ultig ist, verfahren sie in diesem Fall analog zu ung"ultigen Zahlen

\item Testen sie \texttt{public Integer calculate(String operation, Integer operand0, Interger operand1)}

\item Schreiben sie eine Methode \texttt{public Integer calculate(String[] calculation)}\\
\textit{calculation} enthalte Zahlen und Operationen in der gewohnten Reichenfolge, also zB 5 + 4 - 3\\
Verwenden sie \texttt{public Integer toInt(String number)} und \texttt{public Integer calculate(String operation, Integer operand0, Interger operand1)}

\item Schreiben sie eine \textit{main}-Methode, die \texttt{public Integer calculate(String[] calculation)} mit args als Parameter aufruft

\end{enumerate}

\begin{center}
\textbf{\large Fertig!}
\end{center}
\end{document}
