\documentclass{beamer}
\usepackage{ngerman}
\usepackage{listings}
\usepackage {ulem}
\usetheme{Madrid}
\title{Programmieren Tutorium}
\author{Florian Tobias Schandinat}
\date{10.1.2010}
\institute{FTS}
\definecolor{mygreen}{RGB}{0,128,0}
\definecolor{mybrown}{RGB}{128,0,0}
\definecolor{mygray}{RGB}{240,240,240}
\lstset{language=Java,
	backgroundcolor=\color{mygray},
	basicstyle=\tiny\ttfamily,
	keywordstyle=\color{blue},
	stringstyle=\color{mybrown},
	commentstyle=\color{gray},
	numbers=left,
	numberstyle=\color{mygreen},
	frame=single,
	tabsize=4,
	showstringspaces=false,
	xleftmargin=3.5em,Smilie Trauer
	xrightmargin=2em}


\begin{document}


\begin{frame}
\frametitle{Willkommen}
\pause
\begin{alertblock}{Letztes Mal}
\begin{itemize}
\item Vererbung
\item Rekursion
\end{itemize}
\end{alertblock}

\begin{block}{Dieses Mal}
\begin{itemize}
\item Modellierung
\end{itemize}
\end{block}
\end{frame}


\begin{frame}
\begin{center}
\textbf{\Huge Ein erfolgreiches und gesundes neues Jahr!}
\end{center}
\end{frame}


\begin{frame}
\frametitle{R"uckblick: 4. "Ubungsblatt}
\begin{block}{Anmerkungen}\pause
\begin{itemize}
\item Tut mir leid, dass ihr keine Chance hattet aus euren Fehlern zu lernen\pause
\item \alert{Autsch, autsch, autsch!}\pause
\item Von Referenzvergleich, "uber Endlosrekursion zum Exceptionmassaker\pause
\item Bei den Abschlussaufgaben konnten wir in Funktionalit"at (1 von 2 Kategorien) nur noch bestenfalls eine 3 erreichen, wenn auch nur eine Exception fliegt!\pause
\item Konzentriert euch doch bitte, wenn ihr die "Ubungen macht:\\Wenn ihr es an einer Stelle richtig und einer anderen falsch macht kommt es gar nicht gut!\pause
\item Mindestqualit"at: Praktomat gl"ucklich machen
\item F"ur bessere Noten: Eigene (vollst"andige) Testsuite benutzen
\end{itemize}
\end{block}
\end{frame}


\begin{frame}
\frametitle{Modellierung}
\begin{block}{Was bedeutet das?}
Man soll dar"uber nachdenken, wie man Klassen und Methoden strukturiert, bevor man sie implementiert!\\\pause
Daf"ur identifiziert man m"ogliche Klassen und Methoden aus der Aufgaben-/Problemstellung und arbeitet deren Zusammenh"ange und Attribute heraus\\\pause
Es sollte eigentlich nicht passieren, dass man erst w"ahrend der Implementierung merkt, dass man etwas braucht (\alert{Fehlerquelle})
\end{block}
\end{frame}


\begin{frame}
\frametitle{"Ubung: Modellierung}
\begin{center}
\textbf{\Huge Aufgabe 2: Kniffel}
\end{center}
\end{frame}


\begin{frame}
\frametitle{"Ubung: Fast-Food-Rekursion}
\begin{center}
\textbf{\Huge Aufgabe 4-6}
\end{center}
\end{frame}


\begin{frame}
\frametitle{"Ubung}
\begin{center}
\textbf{\Huge Aufgabe 1:\\Die T"urme von Hanoi}
\end{center}
\end{frame}


\begin{frame}
\frametitle{Hilfe}
\begin{block}{Vertiefungstutorium}
\begin{description}
\item[Wann?] Dienstags, 9:45-11:15
\item[Wo?] 50.34, Raum 010
\end{description}
\textbf{Auch dort sollt ihr die Aufgaben m"oglichst selbstst"andig l"osen!}
\alert{Soweit m"oglich bitte Notebooks mitbringen!}
\end{block}

\begin{block}{Programmierberatung}
\begin{description}
\item[Wann?]
\begin{itemize}
\item Montags, 11:30 - 13:00
\item Dienstags, 9:45 - 11:15
\item Donnerstags, 9:45 - 11:15
\end{itemize}
\item[Wo?] 50.34, Raum -143
\end{description}
\end{block}
\end{frame}


\begin{frame}
\frametitle{Ende}
\begin{block}{TODO}
\begin{enumerate}
\item Anmelden zu den Abschlussaufgaben auf https://studium.kit.edu/ bis \alert{23.1.2011}
\item Einreichen einer L"osung f"ur das 6. "Ubungsblatt im Praktomat bis \alert{24.1.2011, 13:00}
\item Anmelden f"ur den "Ubungsschein auf https://studium.kit.edu/ bis \alert{31.3.2011}
\end{enumerate}
\end{block}

\begin{block}{Vielen Dank f"ur die Aufmerksamkeit!}
...und viel Spa"s beim Programmieren :)
\end{block}
\end{frame}


\end{document}

