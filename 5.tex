\documentclass{beamer}
\usepackage{ngerman}
\usepackage{listings}
\usepackage {ulem}
\usetheme{Madrid}
\title{Programmieren Tutorium}
\author{Florian Tobias Schandinat}
\date{22.11.2010}
\institute{FTS}
\definecolor{mygreen}{RGB}{0,128,0}
\definecolor{mybrown}{RGB}{128,0,0}
\definecolor{mygray}{RGB}{240,240,240}
\lstset{language=Java,
	backgroundcolor=\color{mygray},
	basicstyle=\tiny\ttfamily,
	keywordstyle=\color{blue},
	stringstyle=\color{mybrown},
	commentstyle=\color{gray},
	numbers=left,
	numberstyle=\color{mygreen},
	frame=single,
	tabsize=4,
	showstringspaces=false,
	xleftmargin=3.5em,Smilie Trauer
	xrightmargin=2em}


\begin{document}


\begin{frame}
\frametitle{Willkommen}
\pause
\begin{alertblock}{Letztes Mal}
\begin{itemize}
\item Verzweigung
\item Schleifen
\item Tests
\end{itemize}
\pause
\begin{center}
\textbf{\alert{Leider nicht!}}
\end{center}
\end{alertblock}

\pause

\begin{block}{Dieses Mal}
\begin{itemize}
\item Verzweigung
\item Schleifen
\item Tests
\item Arrays
\end{itemize}
\end{block}
\end{frame}


\begin{frame}
\frametitle{R"uckblick :(}
\begin{block}{Wer "ubt auch au"serhalb des Vorlesungs- und "Ubungsbetriebs?}
\pause
$\Rightarrow$ \textbf{Ihr sollt und m"usst ggf. den Stoff nacharbeiten!}\\[0.5em]
\pause
Ihr wollt doch auch nicht, dass der Stoff f"ur das "Ubungsblatt erst \alert{nach} der Abgabe des "Ubungsblattes besprochen wird!
\end{block}

\pause

\begin{block}{Vertiefungstutorium}
\begin{description}
\item[Wann?] Dienstags, 9:45-11:15
\item[Wo?] morgen: 50.34, Raum 148
\end{description}
\textbf{Auch dort sollt ihr die Aufgaben m"oglichst selbstst"andig l"osen}
\end{block}
\end{frame}


\begin{frame}[containsverbatim]
\frametitle{Verzweigung}
\begin{block}{\texttt{if (A) \{B\} else \{C\}}}
Wenn \texttt{A} (Boolscher Ausdruck) zu \texttt{true} evaluiert, wird der Codeblock \texttt{B} ausgef"uhrt, andernfalls wird der Codeblock \texttt{C} ausgef"uhrt\\
\textbf{Hinweis:} Das \texttt{else \{C\}} ist optional
\end{block}

\begin{lstlisting}
public static double max(double a, double b) {
	double max;

	if (a > b) {
		max = a;
	} else {
		max = b;
	}

	return max;
}
\end{lstlisting}
\end{frame}


\begin{frame}[containsverbatim]
\frametitle{Noch mehr Verzweigung}
\begin{block}{Kaskadierung}
\begin{lstlisting}
if (A1) {
	...
} else if (A2) {
	...
} else {
	...
}
\end{lstlisting}
\end{block}

\begin{block}{Verschachtelung}
\begin{lstlisting}
if (A) {
	if (A1) {
		...
	} else {
		...
	}
} else {
	...
}
\end{lstlisting}
\end{block}
\end{frame}


\begin{frame}[containsverbatim]
\frametitle{Schleifen -- Warum?}
\begin{lstlisting}
/* Summiere von 1 bis 4 */
int summe = 0;
summe = summe + 1;
summe = summe + 2;
summe = summe + 3;
summe = summe + 4;
\end{lstlisting}

\begin{block}{Schleife statt copy-and-paste}
\begin{lstlisting}
/* Summiere von 1 bis 4 */
int summe = 0;
for (int i = 1; i <= 4; i = i + 1) {
	summe = summe + i;
}
\end{lstlisting}
\end{block}

\begin{block}{Flexibel/Dynamisch zur Laufzeit}
\begin{lstlisting}
/* Summiere von 1 bis n */
int summe = 0;
for (int i = 1; i <= n; i = i + 1) {
	summe = summe + i;
}
\end{lstlisting}
\end{block}
\end{frame}


\begin{frame}
\frametitle{Schleifen -- for}
\begin{block}{\texttt{for (I; C; A) \{B\}}}
Bei der \texttt{for}-Schleife (auch Z"ahlschleife genannt) wir zu Beginn \texttt{I} ausgef"uhrt\\
Vor jedem Schleifendurchlauf wird der Boolsche Ausdruck \texttt{C} evaluiert und nur wenn er \texttt{true} ist, wird die Schleife durchlaufen, ansonsten geht die Programmausf"uhrung unterhalb der Schleife weiter\\
Bei jedem Schleifendurchlauf wird zuerst \texttt{B}, dann \texttt{A} ausgef"uhrt und anschlie"send wieder \texttt{C} evaluiert um zu entscheiden ob die Schleife nochmal durchlaufen werden soll\\[0.5em]
\textbf{Hinweis:} Normalerweise steht die Anzahl der Durchl"aufe schon vor dem ersten Schleifendurchlauf fest
\end{block}
\end{frame}


\begin{frame}[containsverbatim]
\frametitle{Schleifen -- while}
\begin{block}{\texttt{while (C) \{B\}}}
Die \texttt{while} Schleife wird durchlaufen (d.h. \texttt{B} ausgef"uhrt) solange der Boolsche Ausdruck \texttt{C} zu \texttt{true} evaluiert\\[0.5em]
\textbf{Hinweis:} Wenn \texttt{C} von einer Variable abh"angt die in \texttt{B} bei jedem Durchlauf inkrementiert/dekrementiert wird, k"onnte eine \texttt{for}-Schleife besser sein
\end{block}

\begin{block}{Beispiel}
\begin{lstlisting}
/* success could be already false due to previous errors */
while (success) {
	token = nextToken();
	success = token.isValid();
}
\end{lstlisting}
\end{block}
\end{frame}


\begin{frame}[containsverbatim]
\frametitle{Schleifen -- do-while}
\begin{block}{\texttt{do \{B\} while (C)}}
Sehr "ahnlich zur \texttt{while}-Schleife, nur das bei der \texttt{do-while} Schleife \texttt{B} mindestens einmal ausgef"uhrt wird
\end{block}

\begin{block}{Beispiel}
\begin{lstlisting}
/* do the first run regardless of whether success is true or false */
do {
	token = nextToken();
	success = token.isValid();
} while (success);
\end{lstlisting}
\end{block}
\end{frame}


\begin{frame}
\frametitle{Schleifen -- "Ubung}
\begin{center}
\textbf{\Huge Primzahlen: Aufgabe 1, 2}\\[2em]
\textbf{Hinweis}: Formal sind alle Schleifen "aquivalent
\end{center}
\end{frame}


\begin{frame}
\frametitle{Testen (1)}
\begin{block}{Warum "uberhaupt testen?}
\pause
\textbf{\glqq Ein Feature das nicht getestet wurde existiert auch nicht!\grqq} \\[1em]
Erst das Testen macht das spezifizierte Verhalten von Methoden und Klassen verl"asslich!
\end{block}

\pause

\begin{block}{Was testen?}
\begin{itemize}
\item \textbf{Normalfall} : Das Verhalten bei  ''richtigen'' Eingabedaten
\item \textbf{Randf"alle} : "Ubergangsbereiche (Normallfall $\leftrightarrow$ Fehlerfall)
\item \textbf{Spezialf"alle} : Beispiel: 0!
\item \textbf{Fehlerfall} : Das Verhalten bei ''falschen'' Eingabedaten
\end{itemize}
\end{block}
\end{frame}


\begin{frame}
\frametitle{Testen (2)}
\begin{block}{Manuelles Testen}
Ausgabe von Testwerten und Vergleichen mit erwartetem Ergebniss\\[1em]
-- m"uhsam\\
-- fehleranf"allig
\end{block}

\begin{block}{Automatisches Testen}
Programme "uberpr"ufen die Testwerte mit erwartetem Ergebniss\\[1em]
+ ohne Probleme nach jeder "Anderung durchf"uhrbar\\
-- manche Sachen sind schwierig durch Programme zu "uberpr"ufen
\end{block}
\end{frame}


\begin{frame}
\frametitle{Arrays -- Einleitung}
\begin{block}{Was ist ein Array?}
\pause
\begin{itemize}
\item Ein Objekt
\item Eine Ansammlung von Objekten/Werten des gleichen Typs
\end{itemize}
\textbf{Hinweis: Arrays k"onnen also insbesondere selbst wieder Arrays enthalten}
\end{block}
\end{frame}


\begin{frame}[containsverbatim]
\frametitle{Arrays -- Verwendung (1)}
\begin{block}{Deklaration}
\begin{lstlisting}
int[] primzahlen;
int[][] table;
\end{lstlisting}
\end{block}

\begin{block}{Zuweisung}
\begin{lstlisting}
primzahlen = new int[42];
table = new int[2][2];
\end{lstlisting}
\end{block}

\begin{block}{Initialisierung}
\begin{lstlisting}
String[] fehlermeldungen = {"Datei existiert nicht", "Datei ist kaputt"};
\end{lstlisting}
\end{block}
\end{frame}


\begin{frame}[containsverbatim]
\frametitle{Arrays -- Verwendung (2)}
\begin{block}{Zugriff auf i-tes Element}
\begin{lstlisting}
primzahlen[0] = 2;
table[0][0] = 1;
\end{lstlisting}
\textbf{Hinweis:} Ein Array a mit n Elementen, enth"alt nur die Elemente \texttt{a[0]},...,\texttt{a[n-1]}, \alert{nicht} \texttt{a[n]}
\end{block}

\begin{block}{L"ange ermitteln}
\begin{lstlisting}
int length = primzahlen.length;
\end{lstlisting}
\end{block}
\end{frame}


\begin{frame}
\frametitle{"Ubung}
\begin{center}
\textbf{\Huge Arrays und Schleifen}
\end{center}
\end{frame}


\begin{frame}
\frametitle{Ende}
\begin{block}{TODO}
\begin{enumerate}
\item Einreichen einer L"osung f"ur das 3. "Ubungsblatt im Praktomat bis \alert{6.12.2010, 13:00}
\item Anmelden f"ur den "Ubungsschein auf https://studium.kit.edu/ bis \alert{31.3.2011}
\end{enumerate}
\end{block}

\begin{block}{Vielen Dank f"ur die Aufmerksamkeit!}
...und viel Spa"s beim Programmieren :)
\end{block}
\end{frame}


\end{document}
