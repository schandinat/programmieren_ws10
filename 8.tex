\documentclass{beamer}
\usepackage{ngerman}
\usepackage{listings}
\usepackage {ulem}
\usetheme{Madrid}
\title{Programmieren Tutorium}
\author{Florian Tobias Schandinat}
\date{13.12.2010}
\institute{FTS}
\definecolor{mygreen}{RGB}{0,128,0}
\definecolor{mybrown}{RGB}{128,0,0}
\definecolor{mygray}{RGB}{240,240,240}
\lstset{language=Java,
	backgroundcolor=\color{mygray},
	basicstyle=\tiny\ttfamily,
	keywordstyle=\color{blue},
	stringstyle=\color{mybrown},
	commentstyle=\color{gray},
	numbers=left,
	numberstyle=\color{mygreen},
	frame=single,
	tabsize=4,
	showstringspaces=false,
	xleftmargin=3.5em,Smilie Trauer
	xrightmargin=2em}


\begin{document}


\begin{frame}
\frametitle{Willkommen}
\pause
\begin{alertblock}{Letztes Mal}
\begin{itemize}
\item Listen
\end{itemize}
\end{alertblock}

\begin{block}{Dieses Mal}
\begin{itemize}
\item Sortieren
\end{itemize}
\end{block}

\begin{exampleblock}{N"achstes Mal}
\begin{itemize}
\item Klassengenetik (Vererbung)
\end{itemize}
\end{exampleblock}
\end{frame}


\begin{frame}
\frametitle{R"uckblick: 3. "Ubungsblatt}
\begin{center}
\textbf{\Huge :(}
\end{center}
\end{frame}


\begin{frame}
\frametitle{4. "Ubungsblatt}
\begin{center}
\textbf{\Huge Gibt es Fragen zum 4. "Ubungsblatt?}
\end{center}
\end{frame}


\begin{frame}
\frametitle{Listen -- "Ubung}
\begin{center}
\textbf{\Huge Aufgabe: LinkedPointList}
\end{center}
\end{frame}

\begin{frame}
\frametitle{Sortieren}
\begin{block}{Sortieralgorithmen}
\pause
\begin{itemize}
\item Selectionsort
\item Insertionsort
\item Bubblesort
\end{itemize}
\end{block}

\begin{block}{Bubblesort}
\pause
Eine Liste mit n Elementen\\
Gehe die Liste vom 1. bis zum letzten Element durch und vertausche ein Element mit dem darauffolgenden, wenn es das gr"o"sere von beiden ist\\
Wiederhole dies f"ur die Liste, die aus den ersten n-1 Elementen besteht
\end{block}
\end{frame}


\begin{frame}
\frametitle{Listen -- "Ubung}
\begin{center}
\textbf{\Huge Aufgabe: LinkedPointList}
\end{center}
\end{frame}


\begin{frame}
\frametitle{Hilfe}
\begin{block}{Vertiefungstutorium}
\begin{description}
\item[Wann?] Dienstags, 9:45-11:15
\item[Wo?] 50.34, Raum 010
\end{description}
\textbf{Auch dort sollt ihr die Aufgaben m"oglichst selbstst"andig l"osen!}
\alert{Soweit m"oglich bitte Notebooks mitbringen!}
\end{block}

\begin{block}{Programmierberatung}
\begin{description}
\item[Wann?]
\begin{itemize}
\item Montags, 11:30 - 13:00
\item Dienstags, 9:45 - 11:15
\item Donnerstags, 9:45 - 11:15
\end{itemize}
\item[Wo?] 50.34, Raum -143
\end{description}
\end{block}
\end{frame}


\begin{frame}
\frametitle{Ende}
\begin{block}{TODO}
\begin{enumerate}
\item Einreichen einer L"osung f"ur das 4. "Ubungsblatt im Praktomat bis \alert{20.12.2010, 13:00}
\item Anmelden f"ur den "Ubungsschein auf https://studium.kit.edu/ bis \alert{31.3.2011}
\end{enumerate}
\end{block}

\begin{block}{Vielen Dank f"ur die Aufmerksamkeit!}
...und viel Spa"s beim Programmieren :)
\end{block}
\end{frame}


\end{document}

