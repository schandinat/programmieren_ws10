\documentclass{beamer}
\usepackage{ngerman}
\usepackage{listings}
\usepackage {ulem}
\usetheme{Madrid}
\title{Programmieren Tutorium}
\author{Florian Tobias Schandinat}
\date{20.12.2010}
\institute{FTS}
\definecolor{mygreen}{RGB}{0,128,0}
\definecolor{mybrown}{RGB}{128,0,0}
\definecolor{mygray}{RGB}{240,240,240}
\lstset{language=Java,
	backgroundcolor=\color{mygray},
	basicstyle=\tiny\ttfamily,
	keywordstyle=\color{blue},
	stringstyle=\color{mybrown},
	commentstyle=\color{gray},
	numbers=left,
	numberstyle=\color{mygreen},
	frame=single,
	tabsize=4,
	showstringspaces=false,
	xleftmargin=3.5em,Smilie Trauer
	xrightmargin=2em}


\begin{document}


\begin{frame}
\frametitle{Willkommen}
\pause
\begin{alertblock}{Letztes Mal}
\begin{itemize}
\item Sortieren
\end{itemize}
\end{alertblock}

\begin{block}{Dieses Mal}
\begin{itemize}
\item Vererbung
\item Rekursion
\end{itemize}
\end{block}
\end{frame}


\begin{frame}
\frametitle{4. "Ubungsblatt}
\begin{center}
\textbf{\Huge Wie lief das 4. "Ubungsblatt?}
\end{center}
\end{frame}


\begin{frame}
\frametitle{Nachtrag: 3. "Ubungsblatt}
\begin{block}{Wichtig}
\begin{itemize}
\item Randf"alle, {\LARGE Randf"alle}, \alert{\Huge Randf"alle!}
\item Tests sollen alle public Methoden umfassen und mindestens Normal- und Fehlerf"alle enthalten!
\item Wenn Praktomat meckert kann ich nicht zufrieden sein!
\end{itemize}
\end{block}

\pause

\begin{block}{Stubs}\pause
...oder was mache ich, wenn ich eine Aufgabe nicht l"osen konnte\\\pause
Signatur "ubernehmen und so erg"anzen, dass es kompiliert\\
\lstinline|public static String[] tokenize(String text) \{|\\
\hspace{2em}\lstinline|return null;|\\
\lstinline|\}|\\\pause
Vorteil: Erlaubt es davon abh"angende Teilaufgaben zu bearbeiten
\end{block}
\end{frame}


\begin{frame}
\frametitle{Vererbung}
\begin{block}{Was ist Vererbung?}\pause
Vererbung dient der Konkretisierung einer Klasse
\end{block}

\pause

\begin{block}{\texttt{class A extends B}}\pause
Klasse A erbt von Oberklasse B alle \lstinline|public| und \lstinline|protected| Methoden und Attribute\pause\\
A kann die geerbten Methoden "uberschreiben
\end{block}

\pause

\begin{center}
\textbf{\Huge Freie "Ubung}
\end{center}
\end{frame}


\begin{frame}[containsverbatim]
\frametitle{Rekursion}
\begin{block}{Was ist Rekursion?}
Direkter oder indirekt Selbstaufruf einer Methode\\
\alert{Endlosrekursion ($\leftrightarrow$ Endlosschleifen)}
\end{block}

\begin{block}{Beispiel}
\begin{lstlisting}
public static String echo(String message, int count) {
	String echo = message;
	if (count >= 1) {
		echo += echo(message, count - 1);
	}
	return echo;
}
\end{lstlisting}
\end{block}
\end{frame}


\begin{frame}
\frametitle{Rekursion -- "Ubungen (1)}
\begin{block}{"Ubung 1}
Schreiben Sie eine Methode, die die Fakult"at rekursiv berechnet.
\end{block}

\pause

\begin{block}{"Ubung 2}
Schreiben Sie eine Methode, die die Fibonacci-Folge berechnet.
$$f(n) = n\textrm{, falls $n = 0 \vee n = 1$ }$$
$$f(n) = f(n - 1) + f(n - 2)\textrm{, falls $n \ge 2$ }$$
\end{block}
\end{frame}


\begin{frame}
\frametitle{Rekursion -- "Ubungen (2)}
\begin{block}{"Ubung 3}
Schreiben Sie eine Methode, die die Ackermann-Funktion berechnet.
$$A(m,n) = n + 1\textrm{, falls $m = 0$ }$$
$$A(m,n) = A(m - 1, 1)\textrm{, falls $m > 0 \wedge n = 0$ }$$
$$A(m,n) = A(m - 1, A(m, n - 1))\textrm{, falls $m > 0 \wedge n > 0$ }$$
\end{block}
\end{frame}


\begin{frame}
\frametitle{Hilfe}
\begin{block}{Vertiefungstutorium}
\begin{description}
\item[Wann?] Dienstags, 9:45-11:15
\item[Wo?] 50.34, Raum 010
\end{description}
\textbf{Auch dort sollt ihr die Aufgaben m"oglichst selbstst"andig l"osen!}
\alert{Soweit m"oglich bitte Notebooks mitbringen!}
\end{block}

\begin{block}{Programmierberatung}
\begin{description}
\item[Wann?]
\begin{itemize}
\item Montags, 11:30 - 13:00
\item Dienstags, 9:45 - 11:15
\item Donnerstags, 9:45 - 11:15
\end{itemize}
\item[Wo?] 50.34, Raum -143
\end{description}
\end{block}
\end{frame}


\begin{frame}
\frametitle{Ende}
\begin{block}{TODO}
\begin{enumerate}
\item Einreichen einer L"osung f"ur das 5. "Ubungsblatt im Praktomat bis \alert{6.1.2011, 13:00}
\item Anmelden zu den Abschlussaufgaben auf https://studium.kit.edu/ bis \alert{23.1.2011}
\item Anmelden f"ur den "Ubungsschein auf https://studium.kit.edu/ bis \alert{31.3.2011}
\end{enumerate}
\end{block}

\begin{block}{Vielen Dank f"ur die Aufmerksamkeit!}
Sch"one Weihnachten und einen guten Rutsch ins neue Jahr!\\
...und viel Spa"s beim Programmieren :)
\end{block}
\end{frame}


\end{document}

