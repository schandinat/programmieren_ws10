\documentclass{beamer}
\usepackage{ngerman}
\usepackage{listings}
\usepackage {ulem}
\usetheme{Madrid}
\title{Programmieren Tutorium}
\author{Florian Tobias Schandinat}
\date{17.1.2011}
\institute{FTS}
\definecolor{mygreen}{RGB}{0,128,0}
\definecolor{mybrown}{RGB}{128,0,0}
\definecolor{mygray}{RGB}{240,240,240}
\lstset{language=Java,
	backgroundcolor=\color{mygray},
	basicstyle=\tiny\ttfamily,
	keywordstyle=\color{blue},
	stringstyle=\color{mybrown},
	commentstyle=\color{gray},
	numbers=left,
	numberstyle=\color{mygreen},
	frame=single,
	tabsize=4,
	showstringspaces=false,
	xleftmargin=3.5em,Smilie Trauer
	xrightmargin=2em}


\begin{document}


\begin{frame}
\frametitle{Willkommen}
\pause
\begin{alertblock}{Letztes Mal}
\begin{itemize}
\item Modellierung
\end{itemize}
\end{alertblock}

\begin{block}{Dieses Mal}
\begin{itemize}
\item Exceptions
\item Interfaces
\end{itemize}
\end{block}
\end{frame}


\begin{frame}
\frametitle{Exceptions}
\begin{block}{Was sind \texttt{Exception}s?}\pause
\begin{itemize}
\item \alert{Ausnahmen}, nicht mit regul"arer Fehlerbehandlung verwechseln!\pause
\item Unterklasse(n) von \texttt{Throwable}
\end{itemize}
\end{block}

\pause

\begin{block}{Wie benutzt man \texttt{Exception}s?}\pause
\lstinline|throw new Exception();|\\
\lstinline|try \{ ... \} catch (Exception e) \{ ... \}|\\
Regeln:\pause
\begin{itemize}
\item \texttt{Exception} immer so genau wie m"oglich werfen und fangen (\texttt{NumberFormatException} statt \texttt{Exception})
\item \texttt{try} und \texttt{catch} Codebl"ocke immer so kurz wie m"oglich
\end{itemize}
\end{block}
\end{frame}


\begin{frame}[containsverbatim]
\frametitle{Exceptions -- Beispiel}
\begin{block}{\texttt{throw}}
\begin{lstlisting}
public void addFirst(Document doc) {
	if (doc == null) {
		throw new NullPointerException();
	}
}
\end{lstlisting}
\end{block}

\begin{block}{\texttt{try \{\} catch () \{\}}}
\begin{lstlisting}
int number;
boolean valid = false;
while (!valid) {
	String input = Terminal.askString("Zahl eingeben: ");
	try {
		number = Integer.parseInt(input);
		valid = true;
	} catch (NumberFormatException e) {
		System.out.println("\"" + input + "\" ist keine gueltige Zahl!");
	}
}
\end{lstlisting}
\end{block}
\end{frame}


\begin{frame}
\frametitle{Exceptions -- "Ubung}
\begin{center}
\textbf{\Huge Aufgabe 1: Exceptions}
\end{center}
\end{frame}


\begin{frame}
\frametitle{Interfaces}
\begin{block}{Warum Interfaces?}
Interfaces definieren Schnittstellen, die von mehreren Klassen implementiert werden k"onnen. Dadurch sind sie leicht \alert{austauschbar}!\\\pause
Im Gegensatz zu Klassen (Mehrfachvererbung in Java nicht m"oglich) ist es ohne Probleme m"oglich, dass eine Klasse 2 oder mehr Interfaces implementiert!
\end{block}
\end{frame}


\begin{frame}
\frametitle{Interfaces -- "Ubung}
\begin{center}
\textbf{\Huge Aufgabe 2: Interfaces\\Aufgabe 3: Comparable}
\end{center}
\end{frame}


\begin{frame}
\frametitle{Hilfe}
\begin{block}{Vertiefungstutorium}
\begin{description}
\item[Wann?] Dienstags, 9:45-11:15
\item[Wo?] 50.34, Raum 010
\end{description}
\textbf{Auch dort sollt ihr die Aufgaben m"oglichst selbstst"andig l"osen!}
\alert{Soweit m"oglich bitte Notebooks mitbringen!}
\end{block}

\begin{block}{Programmierberatung}
\begin{description}
\item[Wann?]
\begin{itemize}
\item Montags, 11:30 - 13:00
\item Dienstags, 9:45 - 11:15
\item Donnerstags, 9:45 - 11:15
\end{itemize}
\item[Wo?] 50.34, Raum -143
\end{description}
\end{block}
\end{frame}


\begin{frame}
\frametitle{Ende}
\begin{block}{TODO}
\begin{enumerate}
\item Anmelden zu den Abschlussaufgaben auf https://studium.kit.edu/ bis \alert{23.1.2011}
\item Einreichen einer L"osung f"ur das 6. "Ubungsblatt im Praktomat bis \alert{24.1.2011, 13:00}
\item Anmelden f"ur den "Ubungsschein auf https://studium.kit.edu/ bis \alert{31.3.2011}
\end{enumerate}
\end{block}

\begin{block}{Vielen Dank f"ur die Aufmerksamkeit!}
...und viel Spa"s beim Programmieren :)
\end{block}
\end{frame}


\end{document}

