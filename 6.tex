\documentclass{beamer}
\usepackage{ngerman}
\usepackage{listings}
\usepackage {ulem}
\usetheme{Madrid}
\title{Programmieren Tutorium}
\author{Florian Tobias Schandinat}
\date{29.11.2010}
\institute{FTS}
\definecolor{mygreen}{RGB}{0,128,0}
\definecolor{mybrown}{RGB}{128,0,0}
\definecolor{mygray}{RGB}{240,240,240}
\lstset{language=Java,
	backgroundcolor=\color{mygray},
	basicstyle=\tiny\ttfamily,
	keywordstyle=\color{blue},
	stringstyle=\color{mybrown},
	commentstyle=\color{gray},
	numbers=left,
	numberstyle=\color{mygreen},
	frame=single,
	tabsize=4,
	showstringspaces=false,
	xleftmargin=3.5em,Smilie Trauer
	xrightmargin=2em}


\begin{document}


\begin{frame}
\frametitle{Willkommen}
\pause
\begin{alertblock}{Letztes Mal}
\begin{itemize}
\item Verzweigung
\item Schleifen
\item Tests
\item Arrays
\end{itemize}
\end{alertblock}

\pause

\begin{block}{Dieses Mal}
\begin{itemize}
\item JavaDoc
\end{itemize}
\end{block}
\end{frame}


\begin{frame}
\frametitle{Fazit: 2. "Ubungsblatt}
\begin{block}{Anmerkungen}
\begin{itemize}
\item \lstinline|String toString()| : Hinweise vom "Ubungsleiter\\
$\rightarrow$ Seht es als Hinweis f"ur die Abschlussaufgaben\pause
\item \lstinline|String getContactInformation()| : Vorgegebene Formate strikt befolgen\pause
\item \lstinline|int daysSince1900()|\pause
\item \lstinline|int getAgeInDays(Date today)|
\end{itemize}
\end{block}

\pause

\begin{block}{Allgemein}
\begin{itemize}
\item Vorsicht mit JavaDoc!
\item Vorgegebene Signaturen strikt befolgen\\
$\rightarrow$ besser gleich abschreiben
\item Parameter nicht ignorieren\\
in der Regel (Ausnahme: \texttt{main}) ist das falsch! 
\end{itemize}
\end{block}
\end{frame}


\begin{frame}
\frametitle{"Ubung}
\begin{center}
\textbf{\Huge Arrays und Schleifen}
\end{center}
\end{frame}


\begin{frame}
\frametitle{JavaDoc -- Einleitung} 
\begin{block}{Was ist JavaDoc?}
\lstinline|/** JavaDoc */|\pause\\
Formale Dokumentation\pause\\
Sie dient dem Anwender einer Klasse und soll die nach au"sen sichtbare Schnittstelle (\texttt{public}) beschreiben\pause\\
$\rightarrow$ jede Klasse, Methode und Attribut, die \texttt{public} deklariert sind, muss mit JavaDoc dokumentiert werden!
\end{block}

\pause

\begin{block}{JavaDoc benutzen}
\texttt{javadoc HelloWorld.java}\\
Erzeugt JavaDoc (HTML Dateien) aus HelloWorld.java (die Klasse muss hierf"ur \texttt{public} sein)\\
$\rightarrow$ index.html mit einem Webbrowser "offnen
\end{block}
\end{frame}


\begin{frame}
\frametitle{JavaDoc -- Selber schreiben!}
\begin{block}{Wie ist JavaDoc aufgebaut?}
\begin{itemize}
\item Der erste Satz ist eine Kurzzusammenfassung
\item Alle folgenden S"atze dienen der genaueren Erl"auterung
\item Umlaute sollten nach M"oglichkeit als HTML geschrieben werden
\item Es gibt Tags
\begin{itemize}
\item @param $\langle$Parameternamen$\rangle$ $\langle$Erkl"arung$\rangle$\\
\alert{Muss} f"ur jeden Parameter einer \texttt{public} Methode vorhanden sein
\item @return $\langle$Erkl"arung$\rangle$\\
\alert{Muss} f"ur jede \texttt{public} Methode mit nicht-\texttt{void} R"uckgabetyp vorhanden sein
\end{itemize}
\end{itemize}
\end{block}
\end{frame}


\begin{frame}
\frametitle{Vertiefungstutorium}
\begin{block}{Vertiefungstutorium}
\begin{description}
\item[Wann?] Dienstags, 9:45-11:15
\item[Wo?] 50.34, Raum 010
\end{description}
\textbf{Auch dort sollt ihr die Aufgaben m"oglichst selbstst"andig l"osen!}
\alert{Soweit m"oglich bitte Notebooks mitbringen!}
\end{block}
\end{frame}


\begin{frame}
\frametitle{Ende}
\begin{block}{TODO}
\begin{enumerate}
\item Einreichen einer L"osung f"ur das 3. "Ubungsblatt im Praktomat bis \alert{6.12.2010, 13:00}
\item Anmelden f"ur den "Ubungsschein auf https://studium.kit.edu/ bis \alert{31.3.2011}
\end{enumerate}
\end{block}

\begin{block}{Vielen Dank f"ur die Aufmerksamkeit!}
...und viel Spa"s beim Programmieren :)
\end{block}
\end{frame}


\end{document}

