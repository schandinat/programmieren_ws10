\documentclass{beamer}
\usepackage{ngerman}
\usepackage{listings}
\usepackage {ulem}
\usetheme{Madrid}
\title{Programmieren Tutorium}
\author{Florian Tobias Schandinat}
\date{2.11.2010}
\institute{FTS}
\definecolor{mygreen}{RGB}{0,128,0}
\definecolor{mybrown}{RGB}{128,0,0}
\definecolor{mygray}{RGB}{240,240,240}
\lstset{language=Java,
	backgroundcolor=\color{mygray},
	basicstyle=\tiny\ttfamily,
	keywordstyle=\color{blue},
	stringstyle=\color{mybrown},
	commentstyle=\color{gray},
	numbers=left,
	numberstyle=\color{mygreen},
	frame=single,
	tabsize=4,
	showstringspaces=false,
	xleftmargin=3.5em,
	xrightmargin=2em}


\begin{document}


\begin{frame}
\frametitle{Willkommen}
\end{frame}


\begin{frame}
\frametitle{R"uckblick: 1. "Ubungsblatt}
\begin{center}
\textbf{\Huge Wie ist das 1. "Ubungsblatt gelaufen?}
\end{center}
\end{frame}


\begin{frame}
\frametitle{Einschub: Hello World! -- Reloaded}
\begin{block}{Warum?}
Wir k"onnen jetzt HelloWorld.java compilieren und ausf"uhren...\pause\\
...nun schauen wir es uns mal ein bisschen genauer an!
\end{block}
\end{frame}


\begin{frame}
\frametitle{Einschub: Hello World! -- main}
\lstinputlisting[title=HelloWorld.java]{examples/HelloWorld.java}
\pause
\begin{block}{\texttt{public static void main(String[] args)}}
\begin{itemize}
\item \texttt{public static} Schl"usselw"orter\\
\item \texttt{void} R"uckgabetyp\\
\item \texttt{main} Methodenname\\
\item \texttt{String[] args} Parameter
\end{itemize}
\end{block}
\end{frame}


\begin{frame}
\frametitle{Einschub: Hello World! -- Schl"usselw"orter}
\begin{block}{\texttt{public}}
\begin{itemize}
\item \textbf{Methoden (allgemein)} Erkl"arung kommt im Laufe des Semesters
\item \textbf{main (speziell)} Muss immer \texttt{public} sein!
\end{itemize}
\end{block}

\pause

\begin{block}{\texttt{static}}
\begin{itemize}
\item \textbf{Methoden (allgemein)} ''normale'' Methoden (ohne \texttt{static}) beziehen sich auf ein konkretes Objekt dessen Attribute sie lesen (und gegebenfalls ver"andern) k"onnen. Im Gegensatz dazu beziehen sich \texttt{static} Methoden auf \alert{kein Objekt} und werden deswegen auch als Klassenmethoden bezeichnet
\item \textbf{main (speziell)} Muss immer \texttt{static} sein! (kein Objekt vorhanden)
\end{itemize}
\end{block}
\end{frame}


\begin{frame}
\frametitle{Einschub: Hello World! -- R"uckgabetyp und Methodenname}
\begin{block}{R"uckgabetyp}
Dient der R"uckgabe von Informationen an den Aufrufer\\
Beispiel: Signalisierung von Fehlern\\
\begin{itemize}
\item \textbf{Methoden (allgemein)} Grunds"atzlich kann jeder Typ, wie wir sie von Attributen kennen, als R"uckgabetyp verwendet werden (also unter anderem auch Klassen). Zus"atzlich gibt es noch \texttt{void}, was bedeutet, dass nichts zur"uckgegeben wird
\item \textbf{main (speziell)} Muss immer \texttt{void} sein! (Festlegung)
\end{itemize}
\end{block}

\pause

\begin{block}{Methodennamen}
\begin{itemize}
\item \textbf{Methoden (allgemein)} Es gilt das gleiche wie f"ur Attributnamen
\item \textbf{main (speziell)} Muss nat"urlich \texttt{main} sein ;)
\end{itemize}
\end{block}
\end{frame}


\begin{frame}
\frametitle{Einschub: Hello World! -- Parameter (allgemein)}
\begin{block}{Parameter"ubergabe}
Mit Parametern kann der Aufrufer das Verhalten der Methode beeinflussen\\
Beispiel: Zahl zu einer anderen addieren (Parameter: zu addierende Zahl)
\end{block}

\pause

\begin{block}{Parameter}
Eine Methode kann 0, 1, ... Parameter haben, die jeweils aus Typ (wie die Typen bei Attributen) und Name (auch 'formaler Parameter' genannt) bestehen\\
Der Wert den ein Parameter zur Ausf"uhrung der Methode annimmt nennt man auch 'aktueller Parameter' und kann mittels des 'formalen Parameters' (Parametername) innerhalb der Methode verwendet werden\\
Beispiele:\\
\lstinline|int power(int base, int exponent)|\\
\lstinline|void printError(String message)|
\end{block}
\end{frame}

\begin{frame}
\frametitle{Einschub: Hello World! -- Parameter (speziell)}
\begin{block}{Parameter bei \texttt{main}}
Bis auf den formalen Parameter, der nur innerhalb der Methode eine Rolle spielt, ist es immer \texttt{String[] args}\\
\texttt{args} ist ein Array (sp"ater in der Vorlesung) von Strings, die java mit "ubergeben werden
\end{block}

\pause

\begin{block}{Beispiele}
\ttfamily
\$ java ParameterOut Hallo\\
args[0]: Hallo\\[0.5em]
\$ java ParameterOut Hello World\\
args[0]: Hello\\
args[1]: World\\[0.5em]
\$ java ParameterOut \textrm{''}Dies ist ein Beispiel\textrm{''}\\
args[0]: Dies ist ein Beispiel
\end{block}
\end{frame}


\begin{frame}
\frametitle{Einschub: Hello World! -- Ausgabe}
\begin{block}{\texttt{System.out.println(\textrm{''}Hello World!\textrm{''})}}
\begin{description}
\item[\texttt{System}] Eine Klasse die Systemfunktionalit"at wie Ein- und Ausgabe zur Verf"ugung stellt\\
\item[\texttt{out}] Ein Attribut der Klasse \texttt{System} vom Typ \texttt{PrintStream}\\
\item[\texttt{println}] Methoden der Klasse \texttt{PrintStream} die verschiedene Datentypen ausgeben k"onnen\\ \texttt{print} verursacht anders als \texttt{println} keinen Zeilenvorschub
\end{description}
\alert{Detailierte Informationen findet ihr in der Java API!}
\end{block}
\end{frame}


\begin{frame}
\frametitle{Einschub: Hello World! -- Ende}
\lstinputlisting[title=HelloWorld.java]{examples/HelloWorld.java}
\begin{center}
\textbf{\Huge Habe ich alles erkl"art?}
\end{center}
\end{frame}


\begin{frame}
\frametitle{Variablen}
\begin{block}{}
Wir kennen bereits Attribute in Klassen...\pause\\
...Variablen in Methoden funktionieren fast genau so!\pause\\[0.5em]
Zum Beispiel zum Zwischenspeichern von Ergebnissen
\end{block}

\pause

\begin{block}{Beispiel 1: Deklaration und Zuweisung}
\ttfamily
double mittelwert;\\
mittelwert = 1 / 2.0 * (x0 + x1);
\end{block}

\begin{block}{Beispiel 2: Initialisierung}
\ttfamily
double flaeche = (xRechts - xLinks) * (yUnten - yOben);
\end{block}

\end{frame}


\begin{frame}
\frametitle{Methoden -- Erg"anzungen}
\begin{block}{Bezugsobjekt}
Alle nicht-\texttt{static} Methoden haben ein Bezugsobjekt\\
Man kann auf Attribute des Bezugsobjekts zugreifen, indem man einfach ihren Attributnamen schreibt\\
Eine Referenz auf das Bezugsobjekt ist via \texttt{this} verf"ugbar
\end{block}

\begin{block}{R"uckgabewerte}
Man kann mittels \texttt{return} den R"uckgabewert einer Methode festlegen\\
Der R"uckgabewert sollte nat"urlich zu dem angegeben R"uckgabetyp passen\\
Bei einem \texttt{return} endet auch die Ausf"uhrung der Methode
\end{block}
\end{frame}


\begin{frame}
\frametitle{"Ubung: Statistik}
\begin{block}{Aufgabe}
Schreiben Sie eine Klasse die Mittelwert und Standardabweichung f"ur Datens"atze berechnet. Die Daten werden einzeln in Form von Gleitkommazahlen an eine Methode \texttt{hinzufuegen} ihrer Klasse "ubergeben. Der Mittelwert soll zu jeder Zeit "uber die Methode \texttt{leseMittelwert} und die Standardabweichung "uber eine Methode \texttt{leseStandardabweichung} augelesen werden k"onnen.\\Sie k"onnen hierbei davon ausgehen, dass die Methoden zum Auslesen nicht aufgerufen werden, bevor mindestens 2 Datens"atze eingelesen wurden!
\end{block}

\begin{block}{Hilfe (ohne Gew"ahr)}
\begin{description}[Standardabweichung]
\item[Mittelwert] $E(X) = \frac{1}{n} \underset{i = 1}{\overset{n}{\Sigma}} x_i = \frac{x_1 + x_2 + ... + x_n}{n}$
\item[Varianz] $Var(X) = E(X^2) - (E(X))^2$
\item[Standardabweichung] $\sigma_X = \sqrt{Var(X)}$ \hspace{5em} (Java: \texttt{Math.sqrt})
\end{description}
\end{block}
\end{frame}


\end{document}
