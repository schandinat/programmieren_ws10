\documentclass{beamer}
\usepackage{ngerman}
\usepackage{listings}
\usepackage {ulem}
\usetheme{Madrid}
\title{Programmieren Tutorium}
\author{Florian Tobias Schandinat}
\date{25.10.2010}
\institute{FTS}
\definecolor{mygreen}{RGB}{0,128,0}
\definecolor{mybrown}{RGB}{128,0,0}
\definecolor{mygray}{RGB}{240,240,240}
\lstset{language=Java,
	backgroundcolor=\color{mygray},
	basicstyle=\tiny\ttfamily,
	keywordstyle=\color{blue},
	stringstyle=\color{mybrown},
	commentstyle=\color{gray},
	numbers=left,
	numberstyle=\color{mygreen},
	frame=single,
	tabsize=4,
	showstringspaces=false,
	xleftmargin=3.5em,
	xrightmargin=2em}


\begin{document}


\begin{frame}
\frametitle{Willkommen}
\begin{block}{Programmieren Tutorium 6}
\begin{description}
\item[Wer?] Florian Tobias Schandinat\\
\item[Wo?] 50.34, Raum -108\\
\item[Wann?] jeden Montag 17:30-19:00
\end{description}
\end{block}

\begin{block}{Material online}
http://github.com/schandinat/programmieren\_ws10
\end{block}
\end{frame}


\begin{frame}
\frametitle{Ziel}
\begin{block}{Prim"arziel}
Konzept der objekt-orientierten Programmierung verstehen und es in Form von (Java) Programmen umsetzen k"onnen\\[1em]
Diese Veranstaltung richtet sich auch an StudentInnen, die noch nie in ihrem Leben programmiert haben!
\end{block}

\pause

\begin{block}{Sekund"arziel}
Google (Suchmaschine) verstehen und nachbauen
\end{block}
\end{frame}


\begin{frame}
\frametitle{Organisatorisches}
\begin{block}{"Ubungsbetrieb}
14 t"agig\\
6 "Ubungsbl"atter, insgesamt ca. 120 Punkte\\
Unbenoteter "Ubungsschein: $\ge$ 60 Punkte\\
Anmeldung zum Schein unter https://studium.kit.edu/\\
Abgabe: \alert{Praktomat, montags bis 13:00}\\
https://praktomat.info.uni-karlsruhe.de/
\end{block}

\pause

\begin{block}{Disclaimer}
\alert{Jetzt ausf"ullen und abgeben!}
\end{block}
\end{frame}


\begin{frame}
\frametitle{Praktomat}
\begin{center}
\textbf{\Huge Demo}
\end{center}
\end{frame}


\begin{frame}
\frametitle{Regeln}
\begin{block}{}
\begin{itemize}
\item \alert{Niemand schreibt ab!}\pause\\
\item Im Tutorium redet immer nur eine Person\pause\\
\item Fragen sofort stellen\\Das Tutorium ist daf"ur da um eure Fragen zu beantworten!
\end{itemize}
\end{block}
\end{frame}


\begin{frame}
\frametitle{Fragen}
\begin{center}
\textbf{\Huge Habe ich etwas vergessen?}
\end{center}
\end{frame}


\begin{frame}
\frametitle{Java}
\begin{block}{Von der Quelle bis zur M"undung}
$$\overset{programmieren}{\underset{\langle du\rangle}{\longrightarrow}} \underset{\text{Quellcode}}{\text{.java}} \overset{compilieren/"ubersetzen}{\underset{javac}{\longrightarrow}} \underset{\text{Bytecode}}{\text{.class}} \overset{interpretieren/ausf"uhren}{\underset{java}{\longrightarrow}}$$
\end{block}

\pause

\begin{block}{Hinweise}
\begin{itemize}
\item Quellcode wird in Textdateien (plaintext) gespeichert\\
\item Immer nur eine Klasse pro Datei\\
\item $\langle$Dateiname$\rangle$ = $\langle$Klassenname$\rangle$.java\\
\item Nur Klassen mit einer main-Methode sind ausf"uhrbar
\end{itemize}
\end{block}
\end{frame}


\begin{frame}
\frametitle{Hello World!}
\lstinputlisting[title=HelloWorld.java]{examples/HelloWorld.java}
\pause
\begin{block}{Wie compiliert man das und f"uhrt es ggf. hinterher aus?}
\pause
javac HelloWorld.java\pause\\
java HelloWorld
\end{block}
\pause
\begin{block}{Ausgabe}
Hello World!
\end{block}
\end{frame}


\begin{frame}
\frametitle{Was ist (k)eine Textdatei?}
\begin{block}{.doc, .docx, .odt}
\pause
sind \alert{keine} Textdateien!\\
Sie enthalten neben dem eigentlich Text auch noch Formatierungen und Metainformationen!\\
Unter anderem deshalb programmiert man \alert{nicht} mit Microsoft Word oder Open/Libre Office.
\end{block}

\pause

\begin{block}{.txt, .java}
sind Textdateien.\\
Sie enthalten wirklich nur den angezeigten Text!\\
Man erstellt Sie mit einem Texteditor (n"achste Folie). Zum Programmieren bevorzugt man solche die Syntax-Highlighting unterst"utzen.
\end{block}
\end{frame}

\begin{frame}
\frametitle{Tools}
\begin{block}{Texteditoren}
Windows: Notepad++, Notepad\\
Linux: gedit, Kate\\
Mac: TextEdit, SimpleText\\
...und viele andere, welchen ihr benutzt bleibt euch "uberlassen!
\end{block}

\begin{block}{Java Development Kit (JDK)}
\begin{itemize}
\item \sout{Sun} Oracle JDK: http://java.sun.com/javase/downloads/
\item OpenJDK: http://openjdk.java.net/install/
\end{itemize}
Hinweis: Unter Mac OS ist JDK standardm"a"sig installiert
\end{block}

\begin{block}{Java API - Dokumentation}
http://download.oracle.com/javase/6/docs/api/
\end{block}
\end{frame}


\begin{frame}
\frametitle{Eine kleine Aufgabe f"ur zu Hause}
\begin{block}{Aufgabe}
HelloWorld.java kompilieren und ausf"uhren
\end{block}

\begin{block}{Fragen und Probleme}
gerne auch per E-Mail an FlorianSchandinat@gmx.de
\end{block}
\end{frame}


\begin{frame}
\frametitle{Klasse und Objekt}
\pause
\begin{block}{Klasse}
Abstraktion von Objekten\\
\pause
\begin{itemize}
\item Attribute
\item Methoden
\end{itemize}
\end{block}

\pause

\begin{block}{Objekt}
Instanz einer Klasse\\
\pause
\begin{itemize}
\item Identit"at ($\leftrightarrow$ Instanz)
\item Zustand ($\leftrightarrow$ Attribute)
\item Verhalten ($\leftrightarrow$ Methoden)
\end{itemize}
\end{block}
\end{frame}


\begin{frame}
\frametitle{Attribute}
\begin{block}{Woraus besteht ein Attribut?}
\pause
\begin{itemize}
\item Typ
\item Bezeichner (Attributname)
\end{itemize}
\pause
Beispiele:\\
\lstinline|int anzahl;|\\
\lstinline|int gewicht; // in kg|
\end{block}

\pause

\begin{block}{Typ}
\begin{itemize}
\item Elementare Datentypen
\item Klassen
\end{itemize}
\end{block}
\end{frame}


\begin{frame}
\frametitle{Typ -- Elementare Datentypen}
\begin{block}{Wichtig}
Es werden die Werte direkt gespeichert
\end{block}

\pause

\begin{block}{Welche gibt es und was speichern sie?}
\pause
\begin{itemize}
\item boolean -- Wahrheitswert: true, false\pause\\
\item int -- Ganzzahl: 1, 2, 670, -210\pause\\
\item float -- Kommazahl (einfache Genauigkeit): 1, 0.1f, -0.5f, 1E-10f\pause\\
\item double -- Kommazahl (doppelte Genauigkeit): -0.5, 1E-50\pause\\
\item char -- Einzelnes Zeichen: 'a', 'b', '1', '2', '-', '\#'\\
\item \dots
\end{itemize}
\end{block}
\end{frame}


\begin{frame}
\frametitle{Typ -- Klassen}
\begin{block}{Wichtig}
Es werden lediglich Referenzen auf die Objekte gespeichert
\end{block}

\pause

\begin{block}{Vordefinierte Klassen}
\pause
\begin{itemize}
\item String -- Ganze Zeichenketten: ''Hello World!'', ''-13''\pause\\
\item Integer -- Ganzzahl (analog zu int, kann aber auch \textit{null} sein)\\
\item \dots
\end{itemize}
\end{block}
\end{frame}


\begin{frame}
\frametitle{Attributnamen}
\begin{block}{sollen ... sein}
\begin{itemize}
\item aussagekr"aftig (nicht: \textit{a, b, c, attribut1, attribut2})\\
\item kurz und pr"agnant (keine Romane)\\
\item mit einem kleinen Anfangsbuchstaben (\textit{gewicht} statt \textit{Gewicht})\\
\item bei aus mehreren W"ortern zusammengesetzten Namen die einzelnen Wortanf"ange mit Gro"sbuchstaben markiert (\textit{linkesRad})
\end{itemize}
\end{block}
\end{frame}


\begin{frame}
\frametitle{Kommentare}
\begin{block}{sollen ...}
\begin{itemize}
\item erg"anzende Informationen liefern\\
\item nicht das offensichtliche (Programmiersprache) beschreiben
\end{itemize}
\end{block}

\pause

\begin{block}{3 Arten von Kommentaren}
\begin {itemize}
\item Einzeiliger Kommentar \lstinline|// ...|\\
\item Mehrzeiliger Kommentar \lstinline|/* ... */|\\
\item JavaDoc Kommentar \lstinline|/** ... */|\\
folgt genauen Regeln, mehr dazu ein anderes Mal
\end{itemize}
\end{block}

\pause

\begin{block}{Wichtig}
\alert{Der Code soll auch ohne Kommentare verst"andlich sein!}
\end{block}
\end{frame}


\begin{frame}
\frametitle{"Ubung}
\begin{center}
\textbf{\Huge Auto}
\end{center}
\end{frame}


\begin{frame}
\frametitle{War dieses Tutorium hilfreich?}
\begin{center}
\textbf{\Huge Wer meint, dass er/sie das 1. "Ubungsblatt jetzt im Prinzip l"osen kann?}
\end{center}
\end{frame}


\begin{frame}
\frametitle{Ende}
\begin{block}{TODO}
\begin{enumerate}
\item Einreichen einer L"osung f"ur das 1. "Ubungsblatt im Praktomat bis \alert{1.11.2010, 13:00}
\item Anmelden f"ur den "Ubungsschein auf https://studium.kit.edu/ bis \alert{31.3.2011}
\item HelloWorld.java compilieren und ausf"uhren
\end{enumerate}
\end{block}

\begin{block}{Vielen Dank f"ur die Aufmerksamkeit!}
...und viel Spa"s beim Programmieren :)
\end{block}
\end{frame}


\end{document}
